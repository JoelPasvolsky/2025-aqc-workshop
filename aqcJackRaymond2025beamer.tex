
%\pdfminorversion=4
\pdfminorversion=4
\documentclass[aspectratio=169,final,11pt,forpublic]{beamer} %
% Options:
% [final] Final version.  Removes CONFIDENTIAL DRAFT marking.  Default false.
% [forpublic] Non-confidential?  Uses copyright footer.  Default false.
% [confidential] Confidential?  Uses confidential footer.  Default false.
% [aspectratio=169] or [aspectratio=43]: Widescreen (16:9) or normal size (4:3)?
%      NOTE: Compile twice if you change the aspect ratio, otherwise things won't look right.


\usepackage{listings}
\usetheme{dwave}
\newcommand\cutcontent[1]{}

\title{\ \\ \vspace{.5cm} Quantum simulation with ocean-sdk\vspace{.5cm}}
\date[2025]{13 June, 2025}
\author{Jack Raymond}

\usepackage{braket,tikz,ulem,xmpmulti,siunitx,algorithm,algorithmic}%
\usetikzlibrary{arrows}
\usetikzlibrary{tikzmark,fit,shapes.geometric}

\definecolor{myorange}{rgb}{0.85000,0.32500,0.09800}%
\definecolor{myblue}{rgb}{0.00000,0.44700,0.74100}%
\definecolor{mypurple}{rgb}{0.49400,0.18400,0.55600}%

\newlength{\figurewidth}
\newlength{\figureheight}

%Allows you to change the text width of a single frame.  This is useful for wide slides
%that don't have wide content.  Example:
%\Narrower[3cm]{ Some content, maybe the whole slide (not counting the title) }
%\Narrower{ Default narrowing is 4cm }
\newcommand\Narrower[2][4cm]{%
\makebox[\linewidth][c]{%
  \begin{minipage}{\dimexpr\textwidth-#1\relax}
  \raggedright#2
  \end{minipage}%
  }%
}

%%%%%%%%%%%%%%%%%%%%%%%%%%%%%%%%%%%%%%%%%%%%%%%%%%%%%%%%%%%%%%%%%%%%%%%%%%%%%%%% 
%%%%%%%%%%%%%%%%%%%%%%%%%%%%%%%%%%%%%%%%%%%%%%%%%%%%%%%%%%%%%%%%%%%%%%%%%%%%%%%% 
%%%%%%%%%%%%%%%%%%%%%%%%%%%%%%%%%%%%%%%%%%%%%%%%%%%%%%%%%%%%%%%%%%%%%%%%%%%%%%%% 
\begin{document}
%\maketitle


%\headshotbackground{Jack Raymond}{Technical Lead, Performance Research}
%\begin{frame}\headshotpic[fill=white]{profile.jpeg}%
  
%\end{frame}

%\setblackbackground{0}
\title{Quantum simulation \\with ocean-sdk\\ AQC 2025}
\maketitlering

\setwhitebackground{0}

\setwhitebackground{3}
\begin{frame}\frametitle{\bf In this talk}
  \begin{columns}
    \column{0.5\linewidth}
    \begin{itemize}
    \item A brief introduction to coherent annealing 
    \item Run your own: Landau-Zener and Kibble-Zurek experiments
    \item Code walk through (Q \& A)
    \end{itemize}
    \column{0.5\linewidth}
    \includegraphics[width=0.8\linewidth]{DWave_Advantage_System.png}
  \end{columns}
\end{frame}

\begin{frame}\frametitle{\bf Transverse-field Ising model annealing}
  \begin{columns}
    \column{0.5\linewidth}
    \vspace{-0.2cm}
  \begin{itemize}
    \item Evolve a multi-spin system for time $t_a$, $H(s=t/t_a) = -\Gamma(s) \sum_i \sigma^x_i + \mathcal{J}(s) H_p$  
    \item[1] Program a problem Hamiltonian $H_P = \sum_{ij} J_{ij}\sigma^z_i \sigma^z_j + \sum_i h_i \sigma^z_i$
    \item[2] Prepare a ground state at (large $\Gamma(s=0)$ and small $\mathcal{J}(s=0))$
    \item[3] Evolve to large $\mathcal{J}(s=1)$ and small $\Gamma(s=1)$
    \item[4] Measure in the computational basis
    \item Kadowaki and Nishimori, Phys. Rev. E 58, 5355 (1998)
  \end{itemize}
  \column{0.5\linewidth}
  \includegraphics[width=0.8\linewidth]{Figure4.png}
\end{columns}
\end{frame}

\begin{frame}\frametitle{\bf Four recent papers}
  \begin{columns}
    \column{0.5\linewidth}
    \vspace{-0.2cm}
  \begin{itemize}
    \item[1] Coherent quantum annealing in
      a programmable 2,000 qubit Ising chain, Nature
      Physics (2022)
    \item[2] Quantum critical dynamics in
      a 5,000-qubit programmable spin glass, Nature
      (2023)
    \item[3] Quantum error mitigation in quantum annealing, npj Quantum Information (2025)
    \item[4] Beyond classical computation in quantum simulation, Science (2025)
  \end{itemize}
  \column{0.5\linewidth}
  \includegraphics[width=0.8\linewidth]{supremacy.png}
\end{columns}
\end{frame}


\begin{frame}\frametitle{\bf Landau-Zener dynamics with 16 qubit random models}
  The probability to excite away from the ground state decays exponentially with the annealing time: the exponent determined by the inverse square gap. 
  \begin{columns}
    \column{0.5\linewidth}
    \begin{center}
    \includegraphics[width = 0.5\linewidth]{An-avoided-crossing-and-a-schematic-drawing-of-the-diabatic-fast-and-adiabatic-slow.png}\\
    \includegraphics[width = 0.8\linewidth]{Figure2A_16q.png}\\
    {\small Quantum critical dynamics in
      a 5,000-qubit programmable spin glass, Nature
      (2023)}
    \end{center}
    \column{0.5\linewidth}
    \includegraphics[width = 0.6\linewidth]{Figure2C_16q.png}
    \end{columns}
\end{frame}

\begin{frame}\frametitle{\bf Kibble-Zurek dynamics in 1D}
  Defect rate decays as a power of the annealing time: the power determined by the universality class of the phase transition.
  \begin{columns}
    \column{0.5\linewidth}
    \begin{center}
      \includegraphics[width = \linewidth]{Coherent2000_Figure1a.png}\\
      {\small Coherent quantum annealing in
      a programmable 2,000 qubit Ising chain, Nature
      Physics (2022)}
    \end{center}
    \column{0.5\linewidth}
    \begin{center}
      \includegraphics[width = \linewidth]{Coherent2000_Figure1bc.png}\\
    \end{center}
\end{columns}
\end{frame}

\begin{frame}\frametitle{\bf Implementing experiments (AQC2024)}
  \begin{columns}
    \column{0.5\linewidth}
    \vspace{-0.2cm}
    \begin{itemize}
      \item Fast anneal newly (AQC 2024!) released in all online QPU solvers
        
      \item Allows simple reproduction of earlier results

      \item E.g. Probability of the ground state
  
      \item 3 spin-glass models (different colors):
        16 variables, 52 couplers (+/-1), various gaps

      \item Schedule $\{\Gamma(s), \mathcal{J}(s)\}$ as published on the website; Matching annealing\_time; $J$ modelled == $J$ programmed
        
      \item Lines are exact closed system dynamics, points are single-programming QA data
  \end{itemize}
  \column{0.5\linewidth}
  \begin{center}
  \includegraphics[width=0.8\linewidth]{Figure2C.pdf}\\
  (reproduction of Figure 2c) \\
  {\small Quantum critical dynamics in
  a 5,000-qubit programmable spin glass, Nature
  (2023)}
  \end{center}
\end{columns}
\end{frame}

\begin{frame}\frametitle{\bf Under 20 lines of code, a 1D experiment}
  \begin{center}
    \includegraphics[width=0.7\linewidth]{chain1024code.png}\\
    This Kibble-Zurek (and previous slide Landau-Zener) code, are the subject of the code walk through.
  \end{center}
\end{frame}


\section{Run the code}
\begin{frame}\frametitle{\bf Run the code (locally or codespaces)}
  At github.com/jackraymond/AQC2025workshop \\
  Code > codespace > +\\
  \begin{itemize}
    \item[] dwave setup - -oob  \# Paste the workshop token for Leap access
    \item[] python main.py - - help  \# Show experiment options
    \item[] python main.py  \# Run with defaults
  \end{itemize}

  \begin{itemize}
  \item solver\_name: e.g. Advantage2\_protototype2.6
  \item model: 'Landau-Zener' or 'Kibble-Zurek'
  \item use\_srt: Randomize (sign flip) the computational basis definition.
  \item parallelize\_embedding: embed more than once, as allowed
  \end{itemize}
  For links and further examples see the jupyter notebook:
  physics-experiments-ocean-sdk.ipynb
\end{frame}
  
\begin{frame}\frametitle{\bf Conclusion}

    \begin{center}
      \includegraphics[width=0.8\linewidth]{Demo.png}
    \end{center}
    Use the latest features : Contribute to open source : Experiment\\
    Thanks for attending!
\end{frame}

\end{document}
%%% Local Variables:
%%% mode: latex
%%% TeX-master: t
%%% End:
